\documentclass[a4paper,10pt]{article}

%A Few Useful Packages
\usepackage{marvosym}
\usepackage{fontspec} 					%for loading fonts
\usepackage{xunicode,xltxtra,url,parskip} 	%other packages for formatting
\RequirePackage{color,graphicx}
\usepackage[usenames,dvipsnames]{xcolor}
\usepackage[big]{layaureo} 				%better formatting of the A4 page
% an alternative to Layaureo can be ** \usepackage{fullpage} **
\usepackage{supertabular} 				%for Grades
\usepackage{titlesec}					%custom \section

%Setup hyperref package, and colours for links
\usepackage{hyperref}
\definecolor{linkcolour}{rgb}{0,0.2,0.6}
\hypersetup{colorlinks,breaklinks,urlcolor=linkcolour, linkcolor=linkcolour}

%FONTS
\defaultfontfeatures{Mapping=tex-text}
%\setmainfont[SmallCapsFont = Fontin SmallCaps]{Fontin}
%%% modified for Karol Kozioł for ShareLaTeX use
\setmainfont[
SmallCapsFont = Fontin-SmallCaps.otf,
BoldFont = Fontin-Bold.otf,
ItalicFont = Fontin-Italic.otf
]
%{Fontin.otf}

%%%

%CV Sections inspired by: 
%http://stefano.italians.nl/archives/26
\titleformat{\section}{\Large\scshape\raggedright}{}{0em}{}[\titlerule]
\titlespacing{\section}{0pt}{3pt}{3pt}
%Tweak a bit the top margin
%\addtolength{\voffset}{-1.3cm}

%Italian hyphenation for the word: ''corporations''
\hyphenation{im-pre-se}

%-------------WATERMARK TEST [**not part of a CV**]---------------
\usepackage[absolute]{textpos}

\setlength{\TPHorizModule}{30mm}
\setlength{\TPVertModule}{\TPHorizModule}
\textblockorigin{2mm}{0.65\paperheight}
\setlength{\parindent}{0pt}

%--------------------BEGIN DOCUMENT----------------------
\begin{document}


\pagestyle{empty} % non-numbered pages

\font\fb=''[cmr10]'' %for use with \LaTeX command

%--------------------TITLE-------------
\par{\centering
		{\Huge Paulo Henrique \textsc{da Silva}
	}\bigskip\par}

%--------------------SECTIONS-----------------------------------
%Section: Personal Data
\section{Dados Pessoais}

\begin{tabular}{rl}
    \textsc{Local e Data} & Florianópolis, Brasil  | 24 Janeiro de 2017 \\
    \textsc{Celular:}     & +55 048 996912206\\
    \textsc{Skype:}     & paulohesilva\\
    \textsc{Email:}     & \href{mailto:phs.paulohenriquesilva@gmail.com}{phs.paulohenriquesilva@gmail.com}
\end{tabular}

%Section: Work Experience at the top
\section{Experiência Profissional}
\begin{tabular}{r|p{11cm}}

 \emph{Atual} & Gerente de Projetos na \textsc{Delinea} , Florianópolis - SC\\\textsc{Jan 2017}&\emph{Delinea Tecnologia Educacional}\\&\footnotesize{Responsável pela definição das tecnologias a serem utilizadas nos projeto. Lider da equipe e membro do desenvolvimento. Desenvolvimento de API's rest para disponibilizar os recuros para a equipe mobile.}\\\multicolumn{2}{c}{} \\

 \emph{Atual} & Sócio/Desenvolvedor \textsc{Surf2Day} , Florianópolis - SC\\\textsc{Jan 2017}&\emph{Aplicatico de Surf para localizar o lugar ideal para a prática do surfe com base nas informações do próprio ujsuário.}\\&\footnotesize{Responsável pelo desenvolvimento da API REST que disponibiliza os recuros para o aplicativo mobile.}\\\multicolumn{2}{c}{} \\

 \emph{2013-2017} & Sócio/Arquiteto de Software na \textsc{Catarina Touch} , Florianópolis - SC\\\textsc{Jan 2013}&\emph{Mobile App Studio Trading}\\&\footnotesize{Responsável pela arquiterura dos projetos de backend. Na sua maioria API's REST para integração com aplicações mobile além de lider da equipe e membro do desenvolvimento.}\\\multicolumn{2}{c}{} \\
 
 \textsc{Jan-Dez. 2012} & \textsC{Analista de sistemas/Desenvolvedor na \textsc{Módula Software}, Florianópolis - SC}\\&\footnotesize{Responsável pela análise e desenvolvimento de sistemas J2EE e alguns frameworks como JSF, Spring, PrimeFaces, Jasper, Hibernate, etc.}\\\multicolumn{2}{c}{} \\

\textsc{Fev-Jun. 2012} & Instrutor de Informática na \textsc{CEBRAC}, Florianópolis - SC \\ &\emph{Empresa brasileira de cursos profissionais}\\&\footnotesize{Desenvolver e implementar a disciplina de informática para o curso Administração Completa.}\\\multicolumn{2}{c}{} \\

\textsc{Mar-Nov. 2011} & Desenvolvedor na \textsc{DIGITRO}, Florianópolis - SC \\ &\emph{Empresa nacional de telecomunicações e segmento da segurança pública.}\\&\footnotesize{Desenvolvimento de sistemas web utilizando Java, GWT and Javascript.} \\\multicolumn{2}{c}{} \\

\textsc{2009 - 2011} & Desenvolvedor na \textsc{VH Soluções em T.I}, Florianópolis - SC \\ &\emph{Empresa de desenvolvimento de software ocm foco no seor público.}\\&\footnotesize{Responsável pela análise e desenvolvimento de software utilizando Java e alguns frameworks como JSF, Spring, PrimeFaces, Jasper, Hibernate, etc.}\\\multicolumn{2}{c}{} \\

\textsc{2004 - 2009} & Gerente de T.I na \textsc{KM Comercio de Utlizades Domésticas}, Florianópolis - SC \\&\footnotesize{Suporte aos usuários, aplicações, bem como a manutenção de equipamentos de informática e backup.}\\\multicolumn{2}{c}{} \\ 

\end{tabular}

%Section: Education
\section{Educação}

\begin{tabular}{rl}	

\begin{tabular}{rl}	
\textsc{Dez} 2008/2 a 2014 & Bacharel em \textsc{Ciência da Computação}, UNIVALI, Santa Catarina\\
&\normalsize \textsc{Advisor}: Prof. Michelle da Silva \textsc{Whangam}\\
& Thesis: ``Smart Gateway oara o monitoramento remoto de máquinas industriais''
\end{tabular}

\textsc{Dez} 2016 a 2017 & Pós graduação em \textsc{Engeharia e Qualidade de Software}, UNIVALI, Santa Catarina\\
\end{tabular}

\begin{tabular}{rl}	
\textsc{Dez} 2008/2 a 2014 & Bacharel em \textsc{Ciência da Computação}, UNIVALI, Santa Catarina\\
&\normalsize \textsc{Advisor}: Prof. Michelle da Silva \textsc{Whangam}\\
& Thesis: ``Smart Gateway oara o monitoramento remoto de máquinas industriais''
\end{tabular}

%Section: Scholarships and additional info
\section{Bolsas de estudo}
\begin{tabular}{r|p{11cm}}

\emph{2015/2016} & PaaS for smart machines monitoring and control - (This project also work with angularjs)\\
&\footnotesize{Este projeto visou o desenvolvimento de uma infra-estrutura inovadora para monitoramento e controle remoto de máquinas industriais, combinando as vantagens de IoT e Cloud Computing. O objetivo principal do projeto é aperfeiçoar o processo de monitoramento e controle remoto de máquinas industriais, através de uma PaaS (Plataforma como Serviço) que permite a recepção de dados eo desenvolvimento e implantação de aplicações web para monitoramento e controle de máquinas inteligentes.} \\\multicolumn{2}{c}{} \\

\textsc{2013/2014} & Desenvolvimento de um Smart Gateway Development of a Smart Gateway para o monitoramento de máquinas industriais.\\
&\footnotesize{O projeto visou a utilização de conceitos de IOT para desenvolver um software embarcado em um Beagle Bone Black para monitorar máquinas industriais. Para tanto, foi desenvolvida uma API REST para disponibilizar os dados de funcionamento da máquina.}\\\multicolumn{2}{c}{} \\

\textsc{2011/2012} & Produção de objetos de aprendizagem para o ensino de teste de software\\
&\footnotesize{O projeto propõe uma maneira de auxiliar o teste de software usando objetos de aprendizagem com diferentes níveis de granularidade e focar na reutilização desses objetos em diferentes contextos educacionais, como programas de graduação, treinamento de negócios, certificações, etc.}\\\multicolumn{2}{c}{} \\

\end{tabular}

%Section: Scholarships and additional info
\section{Certificados}
\begin{tabular}{rl}
\textsc{Oracle}  & Oracle Certified Java Programmer 6 \\
\textsc{Voffice/Globalcode} & {\textsc{AngularJS}\\
\textsc{Voffice/Globalcode} & {\textsc{Certified Academy WEB}}\\
\textsc{Fundação Bradesco} & {\textsc{HTML Basic/Advanced}\\
\textsc{Fundação Bradesco} & {\textsc{Javascript}\\
\textsc{FGV} & {\textsc{Fundamentals T.I}\\
\textsc{SOS Computadores} & {\textsc{PHP}\\
\textsc{Agile Code} & {\textsc{Clean Code}\\
\end{tabular}

%Section: Scholarships and additional info
\section{Produção Bibliográfica}
\begin{tabular}{r|p{11cm}}

\textsc{1} & SILVA, P. H.; DOMENECH, M. C. ; RAUTA, L. ; CANDIDO, R. ; WANGHAM, Michelle S. . Controle e Monitoramento Remoto de Máquinas Industriais por meio de Smart Gateway na Web das Coisas. In: Computer on the Beach, 2015, Florianópolis. Computer on the Beach 2015, 2015.\\\\\multicolumn{2}{c}{} \\

\textsc{2} & SILVA, P. H.; BENITTI, F. B. V. ; ALBANO, E. L. . Objetos de aprendizagem para apoio ao ensino de teste de software. In: Computer on the Beach, 2013, Florianópolis. IV Computer on the Beach. p. 314-316.\\\\\multicolumn{2}{c}{} \\

\textsc{3} & SILVA, P. H.; WANGHAM, Michelle S. ; COMUNELLO, E. . Desenvolvimento de um Smart Gateway para o Monitoramento de Máquinas Industriais. In: XIII Seminário de Iniciação Científica, 2014, Itajai. XIII Seminário de Iniciação Científica, 2014.\\\\\multicolumn{2}{c}{} \\

\end{tabular}

%Section: Scholarships and additional info
\section{Informação Adicional}
\begin{tabular}{rl}
\textsc{Set. 2015} & Vonluntário na tradução para português do livro "Python For Informatics".\normalsize\\ &\emph{https://github.com/paulohesilva/py4inf-ptBR}\\
\textsc{Oct. 2015}¨& Primeiro módulo do curso Python Integrated program for everyone.\normalsize\\
&\emph{Universidade de Michigan (Coursera)}\\
\end{tabular}

%Section: Languages
\section{Idiomas}
\begin{tabular}{rl}
 \textsc{Português:}&Nativo\\
\textsc{Inglês:}&Basic\\
\end{tabular}

\end{document}
